% Options for packages loaded elsewhere
\PassOptionsToPackage{unicode}{hyperref}
\PassOptionsToPackage{hyphens}{url}
%
\documentclass[
]{ctexart}
\usepackage{amsmath,amssymb}
\usepackage{lmodern}
\usepackage{iftex}
\ifPDFTeX
  \usepackage[T1]{fontenc}
  \usepackage[utf8]{inputenc}
  \usepackage{textcomp} % provide euro and other symbols
\else % if luatex or xetex
  \usepackage{unicode-math}
  \defaultfontfeatures{Scale=MatchLowercase}
  \defaultfontfeatures[\rmfamily]{Ligatures=TeX,Scale=1}
\fi
% Use upquote if available, for straight quotes in verbatim environments
\IfFileExists{upquote.sty}{\usepackage{upquote}}{}
\IfFileExists{microtype.sty}{% use microtype if available
  \usepackage[]{microtype}
  \UseMicrotypeSet[protrusion]{basicmath} % disable protrusion for tt fonts
}{}
\makeatletter
\@ifundefined{KOMAClassName}{% if non-KOMA class
  \IfFileExists{parskip.sty}{%
    \usepackage{parskip}
  }{% else
    \setlength{\parindent}{0pt}
    \setlength{\parskip}{6pt plus 2pt minus 1pt}}
}{% if KOMA class
  \KOMAoptions{parskip=half}}
\makeatother
\usepackage{xcolor}
\usepackage{color}
\usepackage{fancyvrb}
\newcommand{\VerbBar}{|}
\newcommand{\VERB}{\Verb[commandchars=\\\{\}]}
\DefineVerbatimEnvironment{Highlighting}{Verbatim}{commandchars=\\\{\}}
% Add ',fontsize=\small' for more characters per line
\usepackage{framed}
\definecolor{shadecolor}{RGB}{248,248,248}
\newenvironment{Shaded}{\begin{snugshade}}{\end{snugshade}}
\newcommand{\AlertTok}[1]{\textcolor[rgb]{0.94,0.16,0.16}{#1}}
\newcommand{\AnnotationTok}[1]{\textcolor[rgb]{0.56,0.35,0.01}{\textbf{\textit{#1}}}}
\newcommand{\AttributeTok}[1]{\textcolor[rgb]{0.77,0.63,0.00}{#1}}
\newcommand{\BaseNTok}[1]{\textcolor[rgb]{0.00,0.00,0.81}{#1}}
\newcommand{\BuiltInTok}[1]{#1}
\newcommand{\CharTok}[1]{\textcolor[rgb]{0.31,0.60,0.02}{#1}}
\newcommand{\CommentTok}[1]{\textcolor[rgb]{0.56,0.35,0.01}{\textit{#1}}}
\newcommand{\CommentVarTok}[1]{\textcolor[rgb]{0.56,0.35,0.01}{\textbf{\textit{#1}}}}
\newcommand{\ConstantTok}[1]{\textcolor[rgb]{0.00,0.00,0.00}{#1}}
\newcommand{\ControlFlowTok}[1]{\textcolor[rgb]{0.13,0.29,0.53}{\textbf{#1}}}
\newcommand{\DataTypeTok}[1]{\textcolor[rgb]{0.13,0.29,0.53}{#1}}
\newcommand{\DecValTok}[1]{\textcolor[rgb]{0.00,0.00,0.81}{#1}}
\newcommand{\DocumentationTok}[1]{\textcolor[rgb]{0.56,0.35,0.01}{\textbf{\textit{#1}}}}
\newcommand{\ErrorTok}[1]{\textcolor[rgb]{0.64,0.00,0.00}{\textbf{#1}}}
\newcommand{\ExtensionTok}[1]{#1}
\newcommand{\FloatTok}[1]{\textcolor[rgb]{0.00,0.00,0.81}{#1}}
\newcommand{\FunctionTok}[1]{\textcolor[rgb]{0.00,0.00,0.00}{#1}}
\newcommand{\ImportTok}[1]{#1}
\newcommand{\InformationTok}[1]{\textcolor[rgb]{0.56,0.35,0.01}{\textbf{\textit{#1}}}}
\newcommand{\KeywordTok}[1]{\textcolor[rgb]{0.13,0.29,0.53}{\textbf{#1}}}
\newcommand{\NormalTok}[1]{#1}
\newcommand{\OperatorTok}[1]{\textcolor[rgb]{0.81,0.36,0.00}{\textbf{#1}}}
\newcommand{\OtherTok}[1]{\textcolor[rgb]{0.56,0.35,0.01}{#1}}
\newcommand{\PreprocessorTok}[1]{\textcolor[rgb]{0.56,0.35,0.01}{\textit{#1}}}
\newcommand{\RegionMarkerTok}[1]{#1}
\newcommand{\SpecialCharTok}[1]{\textcolor[rgb]{0.00,0.00,0.00}{#1}}
\newcommand{\SpecialStringTok}[1]{\textcolor[rgb]{0.31,0.60,0.02}{#1}}
\newcommand{\StringTok}[1]{\textcolor[rgb]{0.31,0.60,0.02}{#1}}
\newcommand{\VariableTok}[1]{\textcolor[rgb]{0.00,0.00,0.00}{#1}}
\newcommand{\VerbatimStringTok}[1]{\textcolor[rgb]{0.31,0.60,0.02}{#1}}
\newcommand{\WarningTok}[1]{\textcolor[rgb]{0.56,0.35,0.01}{\textbf{\textit{#1}}}}
\usepackage{graphicx}
\makeatletter
\def\maxwidth{\ifdim\Gin@nat@width>\linewidth\linewidth\else\Gin@nat@width\fi}
\def\maxheight{\ifdim\Gin@nat@height>\textheight\textheight\else\Gin@nat@height\fi}
\makeatother
% Scale images if necessary, so that they will not overflow the page
% margins by default, and it is still possible to overwrite the defaults
% using explicit options in \includegraphics[width, height, ...]{}
\setkeys{Gin}{width=\maxwidth,height=\maxheight,keepaspectratio}
% Set default figure placement to htbp
\makeatletter
\def\fps@figure{htbp}
\makeatother
\setlength{\emergencystretch}{3em} % prevent overfull lines
\providecommand{\tightlist}{%
  \setlength{\itemsep}{0pt}\setlength{\parskip}{0pt}}
\setcounter{secnumdepth}{5}
\ifLuaTeX
  \usepackage{selnolig}  % disable illegal ligatures
\fi
\IfFileExists{bookmark.sty}{\usepackage{bookmark}}{\usepackage{hyperref}}
\IfFileExists{xurl.sty}{\usepackage{xurl}}{} % add URL line breaks if available
\urlstyle{same} % disable monospaced font for URLs
\hypersetup{
  pdftitle={CCA},
  hidelinks,
  pdfcreator={LaTeX via pandoc}}

\title{CCA}
\author{}
\date{\vspace{-2.5em}}

\begin{document}
\maketitle

\begin{Shaded}
\begin{Highlighting}[]
\FunctionTok{library}\NormalTok{(dplyr)}
\end{Highlighting}
\end{Shaded}

\begin{verbatim}
## Warning: package 'dplyr' was built under R version 4.2.3
\end{verbatim}

\begin{verbatim}
## 
## Attaching package: 'dplyr'
\end{verbatim}

\begin{verbatim}
## The following objects are masked from 'package:stats':
## 
##     filter, lag
\end{verbatim}

\begin{verbatim}
## The following objects are masked from 'package:base':
## 
##     intersect, setdiff, setequal, union
\end{verbatim}

\begin{Shaded}
\begin{Highlighting}[]
\FunctionTok{library}\NormalTok{(ggplot2)}
\end{Highlighting}
\end{Shaded}

\begin{verbatim}
## Warning: package 'ggplot2' was built under R version 4.2.2
\end{verbatim}

\begin{Shaded}
\begin{Highlighting}[]
\FunctionTok{library}\NormalTok{(patchwork)}
\end{Highlighting}
\end{Shaded}

\begin{verbatim}
## Warning: package 'patchwork' was built under R version 4.2.3
\end{verbatim}

\begin{Shaded}
\begin{Highlighting}[]
\NormalTok{data }\OtherTok{=} \FunctionTok{read.csv}\NormalTok{(}\StringTok{"data\_2020.csv"}\NormalTok{) }\SpecialCharTok{\%\textgreater{}\%}
  \FunctionTok{select}\NormalTok{(}\SpecialCharTok{{-}}\NormalTok{X)}
\end{Highlighting}
\end{Shaded}

用原数据集而不是降维数据集做CCA,因为要找的是经济因子的线性组合,而主成分之间是相互独立的

\begin{Shaded}
\begin{Highlighting}[]
\CommentTok{\# Construct CCA}
\NormalTok{X }\OtherTok{=} \FunctionTok{as.matrix}\NormalTok{(}\FunctionTok{scale}\NormalTok{(data[,}\DecValTok{4}\SpecialCharTok{:}\DecValTok{12}\NormalTok{]))}
\NormalTok{Y }\OtherTok{=} \FunctionTok{as.matrix}\NormalTok{(}\FunctionTok{scale}\NormalTok{(data[,}\DecValTok{13}\SpecialCharTok{:}\DecValTok{14}\NormalTok{]))}
\NormalTok{CCA }\OtherTok{=}  \FunctionTok{cancor}\NormalTok{(X, Y)}
\NormalTok{w }\OtherTok{=}\NormalTok{ CCA}\SpecialCharTok{$}\NormalTok{xcoef}
\NormalTok{d }\OtherTok{=}\NormalTok{ CCA}\SpecialCharTok{$}\NormalTok{ycoef}
\NormalTok{V1 }\OtherTok{=}\NormalTok{ Y }\SpecialCharTok{\%*\%}\NormalTok{ d[,}\DecValTok{1}\NormalTok{]}
\NormalTok{V2 }\OtherTok{=}\NormalTok{ Y }\SpecialCharTok{\%*\%}\NormalTok{ d[,}\DecValTok{2}\NormalTok{]}
\NormalTok{U1 }\OtherTok{=}\NormalTok{ X }\SpecialCharTok{\%*\%}\NormalTok{ w[,}\DecValTok{1}\NormalTok{]}
\NormalTok{U2 }\OtherTok{=}\NormalTok{ X }\SpecialCharTok{\%*\%}\NormalTok{ w[,}\DecValTok{2}\NormalTok{]}
\NormalTok{rho }\OtherTok{=}\NormalTok{ CCA}\SpecialCharTok{$}\NormalTok{cor}
\end{Highlighting}
\end{Shaded}

First two CC components explain most correlations.

\begin{Shaded}
\begin{Highlighting}[]
\CommentTok{\# First two CC components correlations}
\NormalTok{corr }\OtherTok{=} \FunctionTok{data.frame}\NormalTok{(}\AttributeTok{x =} \FunctionTok{factor}\NormalTok{(}\DecValTok{1}\SpecialCharTok{:}\DecValTok{2}\NormalTok{), }\AttributeTok{y =}\NormalTok{ rho)}
\NormalTok{p1 }\OtherTok{=} \FunctionTok{ggplot}\NormalTok{(corr, }\FunctionTok{aes}\NormalTok{(}\AttributeTok{x =}\NormalTok{ x, }\AttributeTok{y =}\NormalTok{ y)) }\SpecialCharTok{+} \FunctionTok{geom\_col}\NormalTok{(}\AttributeTok{fill =} \StringTok{"steelblue"}\NormalTok{) }\SpecialCharTok{+} 
  \FunctionTok{ggtitle}\NormalTok{(}\StringTok{"Barplot of correlations"}\NormalTok{) }\SpecialCharTok{+}
  \FunctionTok{xlab}\NormalTok{(}\StringTok{"CC component"}\NormalTok{) }\SpecialCharTok{+} \FunctionTok{ylab}\NormalTok{(}\StringTok{"correlation"}\NormalTok{)}
\NormalTok{p1}
\end{Highlighting}
\end{Shaded}

\includegraphics{CCA_files/figure-latex/unnamed-chunk-4-1.pdf}

\begin{Shaded}
\begin{Highlighting}[]
\CommentTok{\# scatterplot of first two CC components}
\NormalTok{df }\OtherTok{=} \FunctionTok{data.frame}\NormalTok{(}\AttributeTok{U =} \FunctionTok{c}\NormalTok{(U1,U2), }\AttributeTok{V =} \FunctionTok{c}\NormalTok{(V1,V2),}
                \AttributeTok{group =} \FunctionTok{as.factor}\NormalTok{(}\FunctionTok{c}\NormalTok{(}\FunctionTok{rep}\NormalTok{(}\DecValTok{1}\NormalTok{,}\FunctionTok{nrow}\NormalTok{(data)),}\FunctionTok{rep}\NormalTok{(}\DecValTok{2}\NormalTok{,}\FunctionTok{nrow}\NormalTok{(data)))))}
\NormalTok{p2 }\OtherTok{=} \FunctionTok{ggplot}\NormalTok{(df, }\FunctionTok{aes}\NormalTok{(}\AttributeTok{x =}\NormalTok{ U, }\AttributeTok{y =}\NormalTok{ V)) }\SpecialCharTok{+} \FunctionTok{geom\_point}\NormalTok{() }\SpecialCharTok{+} \FunctionTok{facet\_wrap}\NormalTok{(}\SpecialCharTok{\textasciitilde{}}\NormalTok{group) }\SpecialCharTok{+}
  \FunctionTok{ggtitle}\NormalTok{(}\StringTok{"Scatter plot of first and second two CC components"}\NormalTok{)}
\NormalTok{p2 }
\end{Highlighting}
\end{Shaded}

\includegraphics{CCA_files/figure-latex/unnamed-chunk-5-1.pdf}

\begin{Shaded}
\begin{Highlighting}[]
\NormalTok{p }\OtherTok{=}\NormalTok{ p1 }\SpecialCharTok{/}\NormalTok{ p2}
\end{Highlighting}
\end{Shaded}

取消注释保存!

\begin{Shaded}
\begin{Highlighting}[]
\NormalTok{p}
\end{Highlighting}
\end{Shaded}

\includegraphics{CCA_files/figure-latex/unnamed-chunk-7-1.pdf}

\begin{Shaded}
\begin{Highlighting}[]
\CommentTok{\#ggsave("CCA.png", p)}
\end{Highlighting}
\end{Shaded}

Here E means economics and H happiness

取消注释保存图片!

\begin{Shaded}
\begin{Highlighting}[]
\CommentTok{\#png("cca\_biplot.png", width = 800, height = 800)}
\NormalTok{name }\OtherTok{=}\NormalTok{ data }\SpecialCharTok{\%\textgreater{}\%} \FunctionTok{select}\NormalTok{(Country) }\SpecialCharTok{\%\textgreater{}\%}
  \FunctionTok{mutate}\NormalTok{(}\AttributeTok{x =} \FunctionTok{paste}\NormalTok{(Country,}\StringTok{"E"}\NormalTok{, }\AttributeTok{sep =} \StringTok{"\_"}\NormalTok{),}
         \AttributeTok{y =} \FunctionTok{paste}\NormalTok{(Country, }\StringTok{"H"}\NormalTok{, }\AttributeTok{sep =} \StringTok{"\_"}\NormalTok{))}
\FunctionTok{biplot}\NormalTok{(}\AttributeTok{x =} \FunctionTok{cbind}\NormalTok{(U1,U2), }\AttributeTok{y =} \FunctionTok{cbind}\NormalTok{(V2,V2),}
       \AttributeTok{xlabs =}\NormalTok{ name}\SpecialCharTok{$}\NormalTok{x, }\AttributeTok{ylabs =}\NormalTok{ name}\SpecialCharTok{$}\NormalTok{y, }\AttributeTok{cex =}\NormalTok{ .}\DecValTok{9}\NormalTok{,}
       \AttributeTok{xlab =} \StringTok{"U1"}\NormalTok{, }\AttributeTok{ylab =} \StringTok{"U2"}\NormalTok{)}
\end{Highlighting}
\end{Shaded}

\includegraphics{CCA_files/figure-latex/unnamed-chunk-8-1.pdf}

PCA and CCA both seek for linear combinations of origina data. However,
the goal of PCA is to explain as much variance as possible, while CCA
tries to perserve the correlations between two data set using linear
combinations. Noticing that here we only perform PCA on economic
indicators instead of the whole numeric data.

\end{document}
